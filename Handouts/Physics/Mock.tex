\documentclass[11pt]{article}
\usepackage[margin=1in]{geometry}
\usepackage[pdftex]{graphicx}
\usepackage{amsmath,amssymb,amsthm}
\usepackage{william}
\usepackage{graphicx} %pictures
\usepackage{wrapfig} %for wrapping pictures
\usepackage{pgfplots} %graphs
\usepackage[makeroom]{cancel} %for striking out math

% code for totaling up points
\usepackage{totcount}
\newtotcounter{tpts}
\newtotcounter{cpts}
\newtotcounter{endpts}
\newcommand{\clearpts}{\addtocounter{tpts}{\value{cpts}} \setcounter{cpts}{0}}
\newcommand{\pts}[1]{\clearpts \setcounter{cpts}{#1}}
\newcommand{\totpts}{\setcounter{endpts}{\totvalue{tpts} + \totvalue{cpts}}\arabic{endpts}}

% formatting for problems and solutions
\definecolor{ptred}{rgb}{0.7,0.1,0.1}
\newcommand{\ptfmt}[1]{\textbf{\color{ptred}#1\color{black}}}
\definecolor{advblue}{rgb}{0.1,0.1,0.7}
\newcommand{\advanced}{\!\textbf{\color{advblue}[A]\color{black}}\ }

\newtheorem*{solution}{Solution}
\newtheorem*{answer}{Answer}

% headers and footers
\usepackage{fancyhdr}
\pagestyle{fancy}
\lhead{\href{https://will-lancer.github.io}{Will Lancer}}
\chead{}
\rhead{USAPhO Mock}
\lfoot{}
\cfoot{\thepage}
\rfoot{}
\renewcommand{\headrulewidth}{0.4pt}
\setlength{\headheight}{14pt}

\newcommand{\bigtitle}[1]{
    \begin{center}
    \huge \textbf{#1}
    \end{center}
}

\makeatletter
\newtheoremstyle{mystyle}
{\topsep}               % space above
{\topsep}               % space below
{}                      % bodyfont
{}                      % indent
{\bfseries}             % headfont
{}                      % punctuation
{0.6em}                 % space after head
{\llap{[\ptfmt{\arabic{cpts}}]\hspace{.6em}}\thmname{#1}\thmnumber{ #2}\thmnote{\normalfont{ (#3)}}{\bfseries .}}  %theoremheadspec
\theoremstyle{mystyle}
\newtheorem{pproblem}{Problem}
\makeatother

% uncomment to hide solutions

% \usepackage{environ}
% \NewEnviron{hide}{}
% \let\solution\hide
% \let\endsolution\endhide
% \let\answer\hide
% \let\endanswer\endhide


\linespread{1.03} % give a little extra room
\setlength{\parindent}{0.2in} % reduce paragraph indent a bit

%%%%%%%%%%%%%%%%%%%%%%%%%%%%%%%%%%%%%%%%%%%%%%

\begin{document}

\bigtitle{USAPhO Mock}

Yep. This is probably going to be a ``super-USAPhO'',
so don't be down if you get like a 2/6. Sources for the
problems are on the last page, plus where you can
find solutions to them. Among us picture for good luck. 
\begin{center}
    \includegraphics[width=0.8\textwidth]{RedAmongUs.jpeg}
\end{center}

\newpage

\section{Oscillating Lake}

\begin{center}
    \includegraphics[width=0.9\textwidth]{mock_lake.png}
\end{center}

The simplest approximation to water sloshing around
in a closed container is that the water's surface
\emph{tilts} while still retaining its flat surface.
Imagine a lake of rectangular cross section, as shown,
of length $L$ and with water depth $h \ll L$. The problem
resembles that of the simple pendulum, in that the kinetic energy is
almost entirely due to horizontal flow of the water, whereas the potential
energy depends on the very small change of vertical level.

\begin{alphamerate}
    \item Imagine at some instant the water level at the
    extreme ends is at $\pm y_0$ with respect to the normal
    water level. Show that the increased gravitational potential
    energy of the whole mass of the water is given by
    \begin{align*}
        U = \frac{1}{6} b \rho g L y_0^2,
    \end{align*}
    where $b$ is the width of the lake.
    \item Assuming the water flow is purely horizontal,
    its speed $v$ must vary with $x$, being greatest at
    $x = 0$ and zero at $x = \pm L/2$. Show that
    the speed of the water as a function of $x$ is
    given by,
    \begin{align*}
        v(x) = v(0) - \frac{x^2}{hL} \frac{dy_0}{dt}.
    \end{align*}
    where,
    \begin{align*}
        v(0) = \frac{L}{4h} \frac{dy_0}{dt}.
    \end{align*}
    You may assume the lake water is incompressible,
    and thus obeys the continuity equation.
    \item Hence show that at any given instant, the total
    kinetic energy of the lake associated with the horizontal
    motion of the water is given by,
    \begin{align*}
        K = \frac{1}{60} \frac{b\rho L^3}{h} \left( \frac{dy_0}{dt} \right)^2.
    \end{align*}
    \item From this, find the period of oscillation of the
    lake. \textbf{Hint:} consider the sum of $K$ and $U$. Does this
    value change with time? You may make reasonable approximations
    if needed.
\end{alphamerate}

\newpage

\section{(Not An) Inverse Square}

Imagine that new and extraordinarily precise measurements 
have revealed an error in Coulomb's law. The \emph{actual} force
of interaction between two point charges is found to be
\begin{align*}
    \mathbf{F} = \frac{q_1 q_2}{4 \pi \epsilon_0} 
    \frac{\mathbf{r} - \mathbf{r}'}{|\mathbf{r} - \mathbf{r}'|^3} 
    \left( 1 + \frac{|\mathbf{r} - \mathbf{r}'|}{\lambda} \right) e^{(|\mathbf{r} - \mathbf{r}'|)/\lambda},
\end{align*}
where $\lambda$ is a new constant of nature called
the PhODs constant. You, the test taker, are now
charged with the task of reformulating electrostatics
to accomodate for the new discovery. Assume
superposition still holds.
\begin{alphamerate}
    \item What is the electric field a distance $\mathbf{r}$
    from a uniform charged spherical shell of radius $\mathbf{a}$,
    with charge density $\sigma$?
    \item Does this version of electrostatics permit a
    scalar potential? Why or why not? (You need only
    give a convincing argument; a proof is not required).
    \item Show that under this formulation of electrostatics,
    not \emph{all} excess charge on a three-dimensional
    conducting object goes to its boundary, only some.
    \item Now suppose we have found \emph{another} error
    in our measurements, and it turns out the new adjustment is
    (completely ignore the previous ``actual'' force for this sub-problem).
    \begin{align*}
        \mathbf{F} \propto r^{-2 + \delta}.
    \end{align*}
    You are going to follow in Maxwell's footsteps, and experimentally
    measure the value of $\delta$. Based on the work of Cavendish, 
    Maxwell used the following experiment to test the value of $\delta$:
    Place two conducting, concentric, thin, spherical shells with radii $a$ and $b$ ($a > b$)
    down, connect a thin wire between them. After charging the outer spherical 
    shell to potential $U$, remove the power supply, then remove the thin 
    wire connecting the two spherical shells, and then ground the outer 
    spherical shell. At this time, the measured potential of the inner 
    spherical shell is not greater than $U$. From this, estimate the value 
    of $\delta$. It is known that $\delta \ll 1$, as it has gone unnoticed
    for so long. 
\end{alphamerate}

\newpage

\section{The Photoelectric Effect}

A zinc ball of radius $R = 1cm$ is located in vacuum,
far away from any other charged bodies, and is charged 
to a potential of $V = -0.5 V$ (assume $V = 0$ at infinity).
The ball is being illuminated by a monochromatic ultraviolet
light with a wavelength of $\lambda = 290nm$.

\begin{alphamerate}
    \item What is the maximum velocity, $v_{\rm max}$ of
    the photoelectrons flying out of the ball?
    \item What is the maximum velocity $v_2$ of a single
    photoelectron very far away from the ball?
    \item Determine the ball's potential after a prolonged
    exposure to the UV light.
    \item Determine the net number, $N$, of photoelectrons
    escaped from the ball after the prolonged exposure to
    the UV light.
\end{alphamerate}

\textbf{Some useful numbers:}
\begin{itemize}
    \item Photoelectric threshold for zinc: $\lambda_0 = 332nm$.
    \item $c = 3 \cdot 10^8 m/s$, $h = 6.626 \cdot 10^{-34} J/s$,
    $\epsilon_0 = 8.85 \cdot 10^{-12} F/m$, $e = -1.6 \cdot 10^{-19}$,
    and the electron mass is $m_e = 9.1 \cdot 10^{-31}kg$.
\end{itemize}

\newpage

\section*{Intermission}

\includegraphics[width=\textwidth]{cat.png}

\newpage


\section{Barrel Stacking}

\begin{center}
    \includegraphics[width=0.5\textwidth]{p4_barrel.png}
\end{center}

Consider the problem of two cylindrical barrels,
one on top of the other, as shown in Figure 6.3. 
The bottom barrel is fixed in position and orientation, 
but the top one, of mass m, is free to move. It starts 
rolling down from its initial position at the top, 
rolling without slipping due to friction between the 
barrels.\\

The bottom barrel has radius $R$, and the top barrel has
radius $a$. $\theta$ is measured from the positive vertical.
The bottom barrel is unable to move, and the top barrel
rolls without slipping on the bottom barrel.
\begin{alphamerate}
    \item Find the height when the top barrel loses
    contact with the bottom barrel.
    \item Find the angle $\theta$ from the vertical
    when the top barrel falls off of the bottom barrel.
    \item Find the distance elapsed on the top barrel
    when the bottom barrel falls off (as in, how
    much circumference of the bottom barrel has
    the top barrel rolled across).
    \item \verb|[***]| Find the normal force as a function of
    $\theta$ (maybe skip this part; I'm pretty sure it's
    harder than all of the other ones combined).
\end{alphamerate}

\newpage

\section{Prism Stacking}

\begin{center}
    \includegraphics[width=0.4\textwidth]{p5_prisms.png}
\end{center}

Two dispersive prisms having apex angles of $A_1 = 60^\circ$
and $A_2 = 30^\circ$ are glued into the shape above.
The refractive indices of the prisms, dependent on
the wavelength, are respectively
\begin{align*}
    n_1(\lambda) & = a_1 + \frac{b_1}{\lambda}\\
    n_2(\lambda) & = a_2 + \frac{b_2}{\lambda}.
\end{align*}
Where $a_1 = 1.1, a_2 = 1.3, b_1 = 10^5 nm^2, b_2 = 5 \cdot 10^4 nm^2$.
\begin{alphamerate}
    \item Determine the wavelength $\lambda_0$ of the incident
    radiation that passes through the prisms without refraction
    on the $AC$ face at any incident angle; determine the corresponding
    refractional indices of the prisms.
    \item Draw the ray path in the system of prisms for three
    different radiations, $\lambda_{\rm red}, \lambda_0, \lambda_{\rm violet}$,
    which are incident on the system at the same angle.
    \item Determine the minimum deviation angle for a ray
    of wavelength $\lambda_0$.
    \item Calculate the wavelength of the ray that penetrates
    and then exits the system parallel to $DC$.
\end{alphamerate}

\newpage

\section{Traveling Faster Than Light}

Can a body move faster than the speed of light? 
The answer is “No” if the object is moving in the vacuum. 
But the answer can be “Yes”, if we deal with the phase speed
of light in an optically dense medium with refractive index of 
$n$ ($n = c/u$, where $u$ is the speed of light in the medium).\\

\noin
We say a body is superluminal if $u < v < c$, where $v$ is
the velocity of the body. Throughout the problem, we will
be dealing with a superluminal body of constant velocity
$v$ in an optical medium without dispersino, with $u$
begin the velocity of light in the medium.\\

\noin
Note that we define $\gamma \equiv 1/\sqrt{1 - (v/c)^2}$,
and $\tan{\theta} = \sqrt{v^2/u^2 - 1}$.


\begin{figure}[h]
    \centering
     \begin{minipage}{0.45\textwidth}
         \centering
         \includegraphics[width=0.9\textwidth]{figure1.png} % first figure itself
    \end{minipage}\hfill
    \begin{minipage}{0.45\textwidth}
         \centering
         \includegraphics[width=0.9\textwidth]{figure3.png} % first figure itself
    \end{minipage}
\end{figure}

\textbf{Note:} there is no figure 2. I had to trim this
question to make it USAPhO-length, so I got rid of the
figure 2 part.

\subsubsection{Figure 1}
For all of these sub-questions, refer to figure 1.
For this figure, a radiating particle is moving along
the $x$-axis with a constant velocity $v$, with $v > u$.
An observer $M$ is located at a distance $d$ from the $x$-axis.
We choose the point closest to the observer on the $x$-axis
to be the origin. The time when the particle passes $x = 0$
is $t = 0$.
\begin{alphamerate}
    \item At time $t = t_0$, the observer first sees the particle at
    position $x_0'$. Find the apparent position $x_0'$ and the observed
    time $t_0$ for this first appearance in terms of $d, v,$ and $\theta$.
    \item Find the apparent position $x'(t)$ at all times $t$. Write
    your answer in terms of $v, \theta, t,$ and $t_0$.
    \item From this, find the apparent velocity, $v'(t)$, at all times $t$.
    Write your answer in terms of $v, \theta, t,$ and $t_0$.
    \item Can an apparent velocity be greater than $c$? Why or why not?
\end{alphamerate}

\subsubsection{Figure 3}
You should refer to figure 3 for all parts of this question.
Assume that the linear radiating object moves perpendicularly
along the $x$-axis. Let the observer be located at the origin,
which we define to be at $x = 0$. The object is symmetric with
respect to the $x$-axis.
\begin{alphamerate}
    \item Show that for a given time $t$, the apparent
    form of this object is an ellipse, or part(s) of an
    ellipse.
    \item Find the position of the center of symmetry for
    the ellipse at a given time $t$, in terms of $v, \theta,$
    and $t$.
    \item Determine the lengths of the semi-major and semi-minor
    axes of the ellipse for a given time $t$, in terms of
    $v, \theta,$ and $t$.
\end{alphamerate}



\newpage


\subsubsection*{Sources}
These problems were inspired/taken from:
\begin{enumerate}
    \item French's \emph{Vibrations and Waves}, problem 3-18
    \item Griffiths EM, problem 2.54, and Guopei, problem idk.
    \item Kiselev and Slobodyanin, page 8, problem 10.
    \item Helliwell and Sahakian's \emph{Modern Classical Mechanics} example 6.2.
    They do it with Lagrangians, but you can do it with Newtonian; it's just
    harder.
    \item IPhO 1983 problem 3
    \item (Some of) APhO 2008 problem 3 (If I included all of
    it, it would have been way too long).
\end{enumerate}

You can get solutions to these problems from
these sources, which you can get by Googling
or by using the resources on the server. For
the Kiselev and Slobodyanin book, I'll post
it in the server or something, along with
Guopei. For the IPhO and APhO problems,
see \href{https://mega.nz/folder/3ZpAGKYJ#hp_Z2CtDlJjhR9shIMHP8w}{this} problem archive.

\subsubsection*{Solutions}
\begin{itemize}
    \item See \href{https://quizlet.com/explanations/textbook-solutions/vibrations-and-waves-1st-edition-9780393099362}{here}.
    \item See \href{https://media.physicsisbeautiful.com/resources/2019/02/18/solutions_manual.pdf}{here}.
    \item See in Discord resources. Ping me and I'll post them.
    \item You can find the book on Libgen or Anna's Archive.
    Not that I'm recommending you do that or anything$\ldots$
    \item See \href{https://mega.nz/folder/3ZpAGKYJ#hp_Z2CtDlJjhR9shIMHP8w}{this} problem archive
    for the $5^{th}$ and $6^{th}$ problem.
\end{itemize}



\end{document}